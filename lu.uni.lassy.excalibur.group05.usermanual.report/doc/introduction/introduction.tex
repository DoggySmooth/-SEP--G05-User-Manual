\chapter{General Information}

\label{chap:introduction}

\section{Scope}
This section has to provide the scope of the user's manual document.
In the following some opening statements to use when providing the
information corresponding to this section.

This document provides the basic knowledge to use SHEAVY \ldots
%Example: This document provides minimum acceptable information for knowing how
% to use the software system \mysystemname.


This document does not explain how it was implemented\ldots 
 
This document is not intended to explain the functionalities behind SHEAVY\ldots
%Example: This document is not intended to provide information about how to
% connect, deploy, configure, or use any external device o
% third-party software system that is rqeuired for the correct funcitoning of
% \mysystemname.

 
This document may be used with \ldots
%This document may be used with other documents provided by third-party
% companies to have an overall view and correct understanding of the environment
% and procedures where the software system \mysystemname is aimed to be deployed
% and run.




\section{Purpose}
In this section you explain the purpose (i.e. aim, objectives) of the user's
manual. In the following some examples of opening statements to be used in this
section.

The purpose of this document is to show the users how to use SHEAVY and
understand it's interface\ldots

This document defines clear usages of SHEAVY \ldots

This document is meant to help the users have their first approach with
the system SHEAVY\ldots



\section{Intended audience}
Description of the categories of persons targeted by this document together with the description of how they are expected to exploit the content of the document.
EXTERIOR: All person which isn't involved with the crisis management will have a
simple guideline to access to the newss published in the application.
INTERIOR: All person activly helping controling the crisis will have a overall
view of the interface. They will know where to find which information. Our
contact list search will be explained to let them find anybody easily, person or
institution.

\section{mysystemname}
Brief overview of the software application domain and main purpose.
Our app is a epidemic crisis management web based project. Data from different
sources will be fetched together to centralize all the known statuses and
information of the crisis in some simple clicks. Allowing to find someone
easily. Realtime view of the ressources locations and distributions.

\subsection{Actors \& Functionalities}
SHeavy has different functionalities for several Actors. Here is an overview.\\

\subsubsection{Common Users}
A common user is defined as an end user who uses SHeavy only in order to
collect information about the possible epidemic and uses the given information
to avoid the infection.
\begin{description}
 \item[$\bullet$] Common users are the principle target of SHeavy. They are
 going to use the application in order to reach information about the possible
 epidemic andfollow the instructions given by Sheavy.
\end{description}

\subsubsection{Coordinators}
The Coordinator is an intermediate between all the Actors and the system. He has
several main functions such as to keep the system functional and updated.
\begin{description}
 \item[$\bullet$] Maintain the system operational: The Coordinator's main task
 is to keep the system operational.
 \item[$\bullet$] If necessary he will perform some improvements and bugs
 corrections on the system and also keep the system maintained and updated.
\end{description} 

\subsection{Operating environment}
Brief overview of the infrastructure on which the software is deployed and used.
Our system is implemented on a securized server from IBM.

\section{Document structure}  
Information on how this document is organised and it is expected to be
used. Recommendations on which members of the audience
should consult which sections of the document, and explanations about the used
notation (i.e. description of formats and conventions) must also be provided.







