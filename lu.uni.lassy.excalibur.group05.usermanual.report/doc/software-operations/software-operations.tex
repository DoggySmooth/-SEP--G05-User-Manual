\chapter{Software operations}
\label{chap:soptware_operations}


Explain each allowed software operations (i.e. an atomic unit of treatment, a service, a functionality) including a brief description of the operation, required parameters, optional parameters, default options, required steps to trigger the operation, assumptions upon request of the operation and expected results of executing such operation.
Describe how to recognise that the operation has successfully been executed or
abnormally terminated. The template given below (i.e. section \ref{operation:MyOperation} has to be used).

Group the operations devoted to the needs of specific actors. Common
operations to several actors may be grouped and presented once to avoid redundancy.


\section{MyOperation}
\label{operation:MyOperation}
The system operator creates and adds a new crisis to the system after being
informed by a third party (citizen, organization) and selects a crisis handler for the crisis.
 
\subsection{MyExample1}
Examples should illustrate the use of \textbf{complex operations}.

Each example must show how the actor uses the software operation under
description to achieve (at least one of) its expected outcome.

It might be required to include GUI screenshots to illustrate the example.\\

\section{Send Alert}
\label{operation:Send Alert}
The Medical Departement creates an alert message to an event, including a
report of what happened and what should be done as a reaction of that event.
He's sending it to the Coordinator who will read it and trigger the alert state.\\
\begin{description}
\item \textbf{Parameters:} Alert Information, User and State Information
\item \textbf{Precondition:} The user is logged in as professional user and is
sending an alert message.
\item \textbf{Post-condition:}  An alert message is sent out to the Coordinator
selected in the alert message window.
\item \textbf{Output messages:} The Coordinator will be notified that an
epidemic is spreading out.
\item \textbf{Triggering:}
\begin{enumerate}
\item In the alert message window, the user fills out the alert information: the title and the description text-fields.
\item The message has to have an report as well as indicating the state of the
alert if needed.
\item Click on the 'Send' button and the Coordinator will be notified of the
epidemic.
\end{enumerate}
\end{description}

\section{Trigger Alert}
\label{operation:TriggerAlert}
The Coordinator recieves an alert message from the Medical Departement
including a report about the severity of the epidemic. The Coordinator will
read it and will trigger the alert and notify every user about the epidemic.\\
\begin{description}
\item \textbf{Parameters:} Alert Information, User and State Information
\item \textbf{Precondition:} The user is logged in as professional user. He
has already read the report and is triggering the alert.
\item \textbf{Post-condition:} The alert is triggered.
\item \textbf{Output messages:} Every user will be notified that an epidemic is
on going.
\item \textbf{Triggering:}
\begin{enumerate}
\item The Coordinator click on the 'Alert' button and the alert is send to very
user.
\end{enumerate}
\end{description}

\section{Find Safe Place}
\label{operation:FindSafePlace}
Any user requests the safest and fastest way to a certain safe place e.g. hospital,
safe camp. The operation will send him through GPS to the selected place.\\
\begin{description}
\item \textbf{Parameters:} Place Information, Place Location
\item \textbf{Precondition:} The user is logged in and is requesting to find the safe place.
\item \textbf{Post-condition:}  The application send the user to the GPS.
\item \textbf{Output messages:} Route to the selected safe place is ready.
\item \textbf{Triggering:}
\begin{enumerate}
\item In the menu the user clicks on 'Safe Place Finder'.
\item A new window shows a list of safe places and the user choses one by 
clicking on the one the user wants to request the route and send the Place Information to SHeavy.
\item SHeavy send back the Place Location which includes the actual location as well as the safest route.
\item The user is lead to the GPS window.
\end{enumerate}
\end{description}

\section{Urgency Call}
\label{operation:UrgencyCall}
Any professional user requests a call to a specific group in a certain location by selecting 
them in the contact list or using the search to find the group (medics, fireman, military) 
by name or location (e.g. hospital/camp name or city they are located).\\
\begin{description}
\item \textbf{Parameters:} Contact Information
\item \textbf{Precondition:} The user is logged in and is requesting to make an urgency call.
\item \textbf{Post-condition:} The application uses call the respective group or person.
\item \textbf{Output messages:} Calling 'ContactX'
\item \textbf{Triggering:}
\begin{enumerate}
\item Open the 'Contacts'-menu.
\item Open the 'Contacts'-menu.
\item Click on 'Call' to request a call to the person or group.
\end{enumerate}
\end{description} 

\section{Send Mission}
\label{operation:SendMission}
The Coordinator send differents mission to various groups leaders (e.g. fire
figther, military, Doctors,..) with some important iformation concerning the mission.\\
\begin{description}
\item \textbf{Parameters:} Contact Information, Mission Information
\item \textbf{Precondition:} The user is logged in as an professional user
and is sending a mission.
\item \textbf{Post-condition:} The mission is recieved by the concerned leader.
\item \textbf{Output messages:} The mission is sent to the leader.
\item \textbf{Triggering:}
\begin{enumerate}
\item The Coordinator opens the 'Mission-view'.
\item Select the concerned leader by slecting a leader in the 'send to' field.
\item Writes the title and select the type by cicking on 'Set Type'.
\item Fill the information about the selected type.
\item Click on the 'Send'-Button to submit the mission.
\end{enumerate}
\end{description} 

\section{Accept Mission}
\label{operation:AcceptMission}
Each of the group's leader can recvieve a mission from he Coordinator. The
leader has to accept or decline the mission if they can't execute it.\\
\begin{description}
\item \textbf{Parameters:} Contact Information, Mission Information
\item \textbf{Precondition:} The user is logged in as an professional user
and recieves the mission notification.
\item \textbf{Post-condition:} The mission will be executed or send to another
team leader.
\item \textbf{Output messages:} The mission is accepted or declined.
\item \textbf{Triggering:}
\begin{enumerate}
\item The team leader recieves a mission.
\item He can click on 'Details' to get further information or decline
immimediately.
\item By clickin ont 'Extras' he get information about other teams present in
the mission zone.
\item He can click on 'accept' or 'decline'
\item After accepting the mission he gets a confirmation notification. He can
quit it by clicking on 'OK'.
\end{enumerate}
\end{description} 

\section{Edit Map}
\label{operation:EditMap}
The Coordinator can edit the map such as create, remove or change zones on the
map.\\
\begin{description}
\item \textbf{Parameters:} Epidemic Information, Change State
\item \textbf{Precondition:} The user is logged in as an professional user
and is creating, removing or changing a zone.
\item \textbf{Post-condition:} The map is updated an ready to use.
\item \textbf{Output messages:} Every users will be notified about the changes.
\item \textbf{Triggering:}
\begin{enumerate}
\item The Coordinator opens the 'Map Editor-view'.
\item He has to click on the 'Add Zone' button to create a new Zone.
\item He has to click on the 'Set Zone' button to change or delete a new Zone.
\item Fill the information about zone to add.
\item Fill the information about zone to change or delete.
\item Click on the 'Add Zone'-Button or 'Set Zone' to submit the modifications
on the map.
\end{enumerate}
\end{description} 

\section{Requirements}
\label{operation:Requirements}
The professional user (medical department, authorities) can
request material needs or support by a resource team. The user has to describe
the needs in the corresponding window.\\
\begin{description}
\item \textbf{Parameters:} Needs Information, Needs Request
\item \textbf{Precondition:} The user has to be logged in as professional and
click on the 'Need' button in the resources menu.
\item \textbf{Post-condition:}  A need request is send to the corresponding groups.
\item \textbf{Output messages:} The needs are send to the corresponding team.
\item \textbf{Triggering:}
\begin{enumerate}
\item Click on menu, then resource and then on needs.
\item The user has to fill out the Needs Information that is the selection of
the receiver, as well as the description of the needs.
\item The user has to indicate his location to add a verification of the needs. 
\item After filling in, click on 'Send' to notify the resource teams of your
request.
\item The user will be notify after the receiver confirms the request. 
\end{enumerate}
\end{description}



