\chapter{Software operations}
\label{chap:soptware_operations}


Explain each allowed software operations (i.e. an atomic unit of treatment, a service, a functionality) including a brief description of the operation, required parameters, optional parameters, default options, required steps to trigger the operation, assumptions upon request of the operation and expected results of executing such operation.
Describe how to recognise that the operation has successfully been executed or
abnormally terminated. The template given below (i.e. section \ref{operation:MyOperation} has to be used).

Group the operations devoted to the needs of specific actors. Common
operations to several actors may be grouped and presented once to avoid redundancy.


\section{MyOperation}
\label{operation:MyOperation}
The system operator creates and adds a new crisis to the system after being
informed by a third party (citizen, organization) and selects a crisis handler for the crisis.
 
\subsection{MyExample1}
Examples should illustrate the use of \textbf{complex operations}.

Each example must show how the actor uses the software operation under
description to achieve (at least one of) its expected outcome.

It might be required to include GUI screenshots to illustrate the example.\\

\section{Requirements}
\label{operation:Requirements}
The professional user (medical department, government) can
request material needs or support by a resource team. The user has to describe
the needs in the corresponding window.\\

\begin{description}

\item \textbf{Parameters:} Needs Information, Needs Request
\item \textbf{Precondition:} The user has to be logged in as professional and
click on the 'Need' button in the resources menu.
\item \textbf{Post-condition:}  A need request is send to the corresponding groups.
\item \textbf{Output messages:} The needs are send to the corresponding team.

\item \textbf{Triggering:}
\begin{enumerate}
\item Click on menu, then resource and then on needs.
\item The user has to fill out the Needs Information that is the selection of
the receiver, as well as the description of the needs.
\item The user has to indicate his location to add a verification of the needs. 
\item After filling in, click on 'Send' to notify the resource teams of your
request.
\item The user will be notify after the receiver confirms the request. 
\end{enumerate}
\end{description}

\section{Send Alert}
\label{operation:Send Alert}
The professional user create an alert message to an event, including a
description of what happened and what should be done as a reaction of that event. \\

\begin{description}

\item \textbf{Parameters:} Alert Information, User and State Information
\item \textbf{Precondition:} The user is logged in as professional user and is sending an alert message.
\item \textbf{Post-condition:}  An alert message is sent out to the users selected in the alert message window.
\item \textbf{Output messages:} The selected users will be notified after the alert is confirmed.

\item \textbf{Triggering:}
\begin{enumerate}
\item In the alert message window, the user fills out the alert information: the title and the description text-fields.
\item Select which user should receive the alert as well as indicating the state of the alert if needed.
\item Click on the 'Send' button and in the next screen click on 'Yes' if you
are sure to submit this message.
\end{enumerate}
\end{description}

\section{Send Alert}
\label{operation:Send Alert}
The professional user create an alert message to an event, including a
description of what happened and what should be done as a reaction of that event. \\

\begin{description}

\item \textbf{Parameters:} Alert Information, User and State Information
\item \textbf{Precondition:} The user is logged in as professional user and is sending an alert message.
\item \textbf{Post-condition:}  An alert message is sent out to the users selected in the alert message window.
\item \textbf{Output messages:} The selected users will be notified after the alert is confirmed.

\item \textbf{Triggering:}
\begin{enumerate}
\item In the alert message window, the user fills out the alert information: the title and the description text-fields.
\item Select which user should receive the alert as well as indicating the state of the alert if needed.
\item Click on the 'Send' button and in the next screen click on 'Yes' if you
are sure to submit this message.
\end{enumerate}
\end{description}

\section{Find Safe Place}
\label{operation:FindSafePlace}
Any user requests the safest and fastest way to a certain safe place e.g. hospital,
safe camp. The operation will send him through GPS to the selected place.\\

\begin{description}

\item \textbf{Parameters:} Place Information, Place Location
\item \textbf{Precondition:} The user is logged in and is requesting to find the safe place.
\item \textbf{Post-condition:}  The application send the user to the GPS.
\item \textbf{Output messages:} Route to the selected safe place is ready.

\item \textbf{Triggering:}
\begin{enumerate}
\item In the menu the user clicks on 'Safe Place Finder'.
\item A new window shows a list of safe places and the user choses one by 
clicking on the one the user wants to request the route and send the Place Information to SHeavy.
\item SHeavy send back the Place Location which includes the actual location as well as the safest route.
\item The user is lead to the GPS window.
\end{enumerate}
\end{description}

\section{Urgency Call}
\label{operation:UrgencyCall}
Any professional user requests a call to a specific group in a certain location by selecting 
them in the contact list or using the search to find the group (medics, fireman, military) 
by name or location (e.g. hospital/camp name or city they are located).\\

\begin{description}

\item \textbf{Parameters:} Contact Information
\item \textbf{Precondition:} The user is logged in and is requesting to make an urgency call.
\item \textbf{Post-condition:} The application uses call the respective group or person.
\item \textbf{Output messages:} Calling 'ContactX'

\item \textbf{Triggering:}
\begin{enumerate}
\item Open the 'Contacts'-menu.
\item Open the 'Contacts'-menu.
\item Click on 'Call' to request a call to the person or group.
\end{enumerate}
\end{description} 



