\chapter{Software operations}
\label{chap:soptware_operations}


Explain each allowed software operations (i.e. an atomic unit of treatment, a service, a functionality) including a brief description of the operation, required parameters, optional parameters, default options, required steps to trigger the operation, assumptions upon request of the operation and expected results of executing such operation.
Describe how to recognise that the operation has successfully been executed or
abnormally terminated. The template given below (i.e. section \ref{operation:MyOperation} has to be used).

Group the operations devoted to the needs of specific actors. Common
operations to several actors may be grouped and presented once to avoid redundancy.


\section{MyOperation}
\label{operation:MyOperation}
The system operator creates and adds a new crisis to the system after being
informed by a third party (citizen, organization) and selects a crisis handler for the crisis.
 
\subsection{MyExample1}
Examples should illustrate the use of \textbf{complex operations}.

Each example must show how the actor uses the software operation under
description to achieve (at least one of) its expected outcome.

It might be required to include GUI screenshots to illustrate the example.\\

\section{Login}
\label{operation:Login}
The coordinator, administrator, leaders or anyone working in the medical
department or authorities wants to access the applications.\\
\begin{description}
\item \textbf{Parameters:} User Information, User Password
\item \textbf{Precondition:} The user has the application open and has already
his credentials.
\item \textbf{Post-condition:}  The user enters the homescreen with access to
his individual information and functions.
\item \textbf{Output messages:} None.
\item \textbf{Triggering:}
\begin{enumerate}
\item After opening the application the user has two textfields for account and
password.
\item The user enters his account and his password.
\item Click on the 'Login' button and the user will be linked to the homescreen.
\end{enumerate}
\end{description}

\section{Add a new User}
\label{operation:AddUser}
In request of the coordinator, the Administrator adds a person as a common user or a professional user 
to the system, allowing them to use the specific operation each type of user has.\\
\begin{description}
\item \textbf{Parameters:} Personal Information, User Information, Coordinator Request
\item \textbf{Precondition:} The crisis didn't has started and the administrator
is logged in. Furthermore, the person who the administrator is going to add
don't has an existing account.
\item \textbf{Post-condition:} The selected user gains access to the specific feature of the system each type of user has.
\item \textbf{Output messages:} The user has been added to the system.
\item \textbf{Triggering:}
\begin{enumerate}
\item The administrator clicks on 'Add User' that opens the Add-User-Interface.
\item The administrator fills in the Personal Information: Title, First Name, Last Name, Phone Number, E-mail.
\item The administrator fills in the requested type of professional from the following: Common User, Medical Department, Authorities and Coordinator.
\item The administrator selects his subtype for the Medical Department and Authorities but not for the Common user and Coordinator.
\item The administrator enters the Users Team.
\item Finally, the administrator clicks on 'Add User'.
\end{enumerate}
\end{description}

\section{Send Alert}
\label{operation:SendAlert}
The Medical Departement creates an alert message for an epidemic, including a
report of what happened and what should be done as a reaction of that epidemic.
He's sending it to the Coordinator who will read it and confirm the alert
state.\\
\begin{description}
\item \textbf{Parameters:} Alert Information, User and State Information
\item \textbf{Precondition:} The user is logged in as professional user such as
Medical Departement.
\item \textbf{Post-condition:}  The alert was successfully sent and the
Coordinator was notified about the alert.
\item \textbf{Output messages:} 'Alert sent'.
\item \textbf{Triggering:}
\begin{enumerate}
\item In the alert message window, the professional user fills out the alert
information: the title and the description text-fields.
\item The message has to have an report as well as indicating the state of the
alert if needed.
\item Click on the 'Send' button and the Coordinator will be notified of the
epidemic.
\end{enumerate}
\end{description}

\section{Confirm Alert}
\label{operation:ConfirmAlert}
The Coordinator recieves an alert message from the Medical Departement
including a report about the severity of the epidemic. The Coordinator will
read it and will confirm the alert and notify every user about the epidemic.\\
\begin{description}
\item \textbf{Parameters:} Alert Information, User and State Information
\item \textbf{Precondition:} The user is logged in as professional user and
has already read the report.
\item \textbf{Post-condition:} Start of the CMS. The alert is confirmed and
every user is notified about he epidemic.
\item \textbf{Output messages:} 'Epidemic confirmed'.
\item \textbf{Triggering:}
\begin{enumerate}
\item The Coordinator recieves an alert notification on his web-interface. He
has to click on 'Report' to read the report sent by the Medical Departement.
\item According to the Coordinator's decision he'll click on 'Launch Crisis' to
start the crisis, 'Delay Decision' to not decide yet if he needs confirmation of
the governement or 'Decline' if the situaton don't need a crisis
management system.
\end{enumerate}
\end{description}

\section{Upgrade User to Professional status}
\label{operation:UpgradeUser}
In request of the coordinator, the Administrator wants to upgrade a common user
to a professional user.\\
\begin{description}
\item \textbf{Parameters:} User name, Coordinator Request
\item \textbf{Precondition:} The administrator has to be on his home screen.
\item \textbf{Post-condition:}  The selected user gains access to the new
feature of the system.
\item \textbf{Output messages:} The user has been modified.
\item \textbf{Triggering:}
\begin{enumerate}
\item The administrator selects or he uses the search to find and select the
requested user in order to modify his data.
\item The administrator selects the requested type of professional from the
following: Common User, Medical Department, Authorities and Coordinator.
\item The administrator selects his subtype for the Medical Department and
Authorities but not for the Common user and Coordinator.
\item The administrator enters the Users Team.
\item The administrator clicks on �Update User� to finish the operation.
\end{enumerate}
\end{description}


\section{Find Safe Place}
\label{operation:FindSafePlace}
Any user requests the safest and fastest way to a certain safe place e.g. hospital,
safe camp. The operation will send him through GPS to the selected place.\\
\begin{description}
\item \textbf{Parameters:} Place Information, Place Location
\item \textbf{Precondition:} The crisis is on going. The user is logged in and
has his GPS enabled.
\item \textbf{Post-condition:}  The fastest way to a certain safe place was
calculated.
\item \textbf{Output messages:} 'Route to the selected safe place is ready.'
\item \textbf{Triggering:}
\begin{enumerate}
\item In the menu the user clicks on 'Safe Place Finder'.
\item A new window shows a list of safe places and the user choses one by 
clicking on the one the user wants to request the route and send the Place Information to SHeavy.
\item SHeavy send back the Place Location which includes the actual location as well as the safest route.
\item The user is lead to the GPS window.
\end{enumerate}
\end{description}

\section{Urgency Call}
\label{operation:UrgencyCall}
Any professional user requests a call to a specific group in a certain location by selecting 
them in the contact list or using the search to find the group (medics, fireman, military) 
by name or location (e.g. hospital/camp name or city they are located).\\
\begin{description}
\item \textbf{Parameters:} Contact Information
\item \textbf{Precondition:} The user is logged in and has to be connected to a
network. The crisis has already started.
\item \textbf{Post-condition:} The application uses call the respective group or person.
\item \textbf{Output messages:} Calling 'ContactX'
\item \textbf{Triggering:}
\begin{enumerate}
\item Open the 'Contacts'-menu.
\item Searchs for the person to call.
\item Click on 'Call' to request a call to the person or group.
\end{enumerate}
\end{description} 

\section{Set Camp Mission}
\label{operation:SetCampMission}
The Coordinator sends set camp mission to various groups leaders (e.g. fire
figther, military, Doctors,..) with some important information concerning the
mission.\\
\begin{description}
\item \textbf{Parameters:} Contact Information, Mission Information
\item \textbf{Precondition:} The user is logged in as a professional user. The
crisis has already started.  
\item \textbf{Post-condition:} The mission is recieved by the concerned leader.
\item \textbf{Output messages:} 'Mission Pending'
\item \textbf{Triggering:}
\begin{enumerate}
\item The Coordinator opens the 'Mission-view'.
\item Select one or more teams by selecting a leader in the 'send to' field or
sends to all by clicking on 'Send to All'.
\item Writes the title and select 'Set Camp' as type by clicking on 'Set
Type'.
\item Give the Location of the camp and also announce a deadline in format Date
and time.
\item Furthermore indicates de capacity of the camp and how many teams are
needed there.
\item Fill the description for more details about the mission.
\item Click on the 'Send'-Button to submit the mission.
\end{enumerate}
\end{description}

\section{Set Checkpoint Mission}
\label{operation:CheckpointMission}
The Coordinator sends set checkpoint mission to various groups leaders (e.g.
fire figther, military, Doctors,..) with some important information concerning the
mission.\\
\begin{description}
\item \textbf{Parameters:} Contact Information, Mission Information
\item \textbf{Precondition:} The user is logged in as a professional user. The
crisis has already started. 
\item \textbf{Post-condition:} The mission is recieved by the concerned leader.
\item \textbf{Output messages:} 'Mission Pending'
\item \textbf{Triggering:}
\begin{enumerate}
\item The Coordinator opens the 'Mission-view'.
\item Select one or more teams by selecting a leader in the 'send to' field or
sends to all by clicking on 'Send to All'.
\item Writes the title and select 'Set Checkpoint' as type by clicking on 'Set
Type'.
\item Give the Location of the camp and also announce a deadline in format Date
and time.
\item Furthermore indicates how many teams are needed there and where victims
are going to be moved.
\item Fill the description for more details about the mission.
\item Click on the 'Send'-Button to submit the mission.
\end{enumerate}
\end{description} 

\section{Transfer Mission}
\label{operation:TransferMission}
The Coordinator sends transfer mission to various groups leaders (e.g. fire
figther, military, Doctors,..) with some important information concerning the
mission.\\
\begin{description}
\item \textbf{Parameters:} Contact Information, Mission Information
\item \textbf{Precondition:} The user is logged in as a professional user. The
crisis has already started. 
\item \textbf{Post-condition:} The mission is recieved by the concerned leader.
\item \textbf{Output messages:} 'Mission Pending'
\item \textbf{Triggering:}
\begin{enumerate}
\item The Coordinator opens the 'Mission-view'.
\item Select one or more teams by selecting a leader in the 'send to' field or
sends to all by clicking on 'Send to All'.
\item Writes the title and select 'Transfer People' as type by clicking on 'Set
Type'.
\item Give the Locations from where to where people are going to be moved.
\item Announce a deadline in format Date and time and how many people are going
to be moved.
\item Click on the 'Send'-Button to submit the mission.
\end{enumerate}
\end{description}

\section{Evacuation Mission}
\label{operation:EvacuateMission}
The Coordinator sends evacuation mission to various groups leaders (e.g. fire
figther, military, Doctors,..) with some important information concerning the
mission.\\
\begin{description}
\item \textbf{Parameters:} Contact Information, Mission Information
\item \textbf{Precondition:} The user is logged in as a professional user. The
crisis has already started.  
\item \textbf{Post-condition:} The mission is recieved by the concerned leader.
\item \textbf{Output messages:} 'Mission Pending'
\item \textbf{Triggering:}
\begin{enumerate}
\item The Coordinator opens the 'Mission-view'.
\item Select one or more teams by selecting a leader in the 'send to' field or
sends to all by clicking on 'Send to All'.
\item Writes the title and select 'Evacuate' as type by clicking on 'Set
Type'.
\item Give the Locations from where to where people are going to be evacuate.
\item Indicate an estimated number of people to be rescued but can be changed
lately.
\item Click on the 'Send'-Button to submit the mission.
\end{enumerate}
\end{description}

\section{Accept Mission}
\label{operation:AcceptMission}
Each group member can recieve a mission from the Coordinator. But only the
leader can accept the mission or decline if they can't execute it.\\
\begin{description}
\item \textbf{Parameters:} Contact Information, Mission Information
\item \textbf{Precondition:} The user is logged in as a professional user
and recieves the mission notification. The crisis has already started. The
maximum number of team needed in the mission doesn't exceed the demande. And the
Deadline isn't reached.
\item \textbf{Post-condition:} The mission will be executed or sent to another
team.
\item \textbf{Output messages:} The mission is Accepted or Transfer.
\item \textbf{Triggering:}
\begin{enumerate}
\item The team leader recieves a mission.
\item He can click on 'Details' to get further information or decline
immimediately.
\item By clickin ont 'Extras' he get information about other teams present in
the mission zone.
\item He can click on 'Accept' or 'Decline'
\item After accepting the mission he gets a confirmation notification. He can
quit it by clicking on 'OK'.
\end{enumerate}
\end{description}  

\section{Add New Zone}
\label{operation:AddNewZoneS}
The Coordinator can add new zones to the map. According to that the map will
be updated.\\
\begin{description}
\item \textbf{Parameters:} Epidemic Information, Change State
\item \textbf{Precondition:} The user is logged in as a professional user. The
crisis has already started. The zone to add doesn't exists.
\item \textbf{Post-condition:} A new zone was added to the map.
\item \textbf{Output messages:} 'Updated'
\item \textbf{Triggering:}
\begin{enumerate}
\item The Coordinator opens the 'Map Editor-view'.
\item He has to click on the 'Add Zone' label to add a new Zone.
\item He has to select a Type of zone by clicking on 'Choose Type'.
\item The Coordinator select the visibility type.
\item Set the area by given an estimation of the height and width in km.
\item At last select the Location where the area will be set.
\item Click on the 'Add Zone'-Button to submit the modifications
on the map.
\end{enumerate}
\end{description} 

\section{Change Zone}
\label{operation:ChangeZone}
The Coordinator can change a zone's propreties in any time. According
to that the map will be updated.\\
\begin{description}
\item \textbf{Parameters:} Epidemic Information, Change State
\item \textbf{Precondition:} The user is logged in as a professional user. The
crisis has already started. The relevant zone exists.
\item \textbf{Post-condition:} An existing zone was modified on the map.
\item \textbf{Output messages:} 'Updated'
\item \textbf{Triggering:}
\begin{enumerate}
\item The Coordinator opens the 'Map Editor-view'.
\item He has to click on the 'Set Zone' label to change a Zone.
\item At first he has to select the zone by clicking on 'Select a Zone' button
or searches for the name of the zone.
\item He can modify the zone's propreties.
\item Click on the 'Safe Changes' to submit the modifications
on the map or click on 'Cancel Changes' to cancel the modification.
\end{enumerate}
\end{description} 

\section{Delete Zone}
\label{operation:DeleteZone}
The Coordinator can delete a zone in any time. According
to that the map will be updated.\\
\begin{description}
\item \textbf{Parameters:} Epidemic Information, Change State
\item \textbf{Precondition:} The user is logged in as a professional user. The
crisis has already started. The Zone exists.
\item \textbf{Post-condition:} An existing zone was deleted on the map.
\item \textbf{Output messages:} 'Updated'
\item \textbf{Triggering:}
\begin{enumerate}
\item The Coordinator opens the 'Map Editor-view'.
\item He has to click on the 'Set Zone' label to change a zone.
\item At first he has to select the zone by clicking on 'Select a Zone' button
or searches for the name of the zone.
\item Click on the 'Delete Zone' button to delete the zone or click on the
'Cancel Change' button to quit.
\end{enumerate}
\end{description} 

\section{Requirements}
\label{operation:Requirements}
The professional user (medical department, authorities) can
request material needs or support by a resource team. The user has to describe
the needs in the corresponding window.\\
\begin{description}
\item \textbf{Parameters:} Needs Information, Needs Request
\item \textbf{Precondition:} The user has to be logged in as professional. The
crisis has already started.
\item \textbf{Post-condition:}  The request is send to the corresponding groups.
\item \textbf{Output messages:} None
\item \textbf{Triggering:}
\begin{enumerate}
\item Click on menu, then resource and then on needs.
\item The user has to fill out the Needs Information that is the selection of
the receiver, as well as the description of the needs.
\item The user has to indicate his location to add a verification of the needs. 
\item After filling in, click on 'Send' to notify the resource teams of your
request.
\item The user will be notify after the receiver confirms the request. 
\end{enumerate}
\end{description}



