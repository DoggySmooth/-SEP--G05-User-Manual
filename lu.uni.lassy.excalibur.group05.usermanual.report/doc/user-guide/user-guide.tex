\chapter{Usage Guide}
\label{chap:usage_guide}

This section is aimed at describing the general use of the software. Such
information is grouped by the different kinds of actors.
Such actors are expected to use the software to perform some
processes or workflows (called here procedures) using the concerned software
\textbf{(including installation procedures)}.

The description of the processes should be organised to facilitate learning by
presenting simpler, more common, or initial processes before more complex, less
utilised, or subsequent processes.

Common procedures should be presented once to avoid redundancy when they are
used in more complex procedures. 

Each process has to be documented using the following use-case textual description
template \cite{armour01usecase} \textbf{BUT its content must be as low level as possible with actual values}:
\vspace{0.5cm}
\hrule
\begin{lyxlist}{UC1}
\small{
\item [\textbf{Use~Case:}] ProcessMissionOne
\item [\textbf{Scope:}] Crisis Management System (\emph{CMS})
\item [\textbf{Primary Actor}:] Coordinator John
\item [\textbf{Secondary Actor}:] FirstAidWorker Bob,\\
                  ExternalResourceSystem (ERS)
\item [\textbf{Intention:}]The intention of the Coordinator is to process mission with ID equal to 1.
\item [\textbf{Level}:]Sub-functional level
\item [\textbf{Main~Success~Scenario}]:\\
1. \emph{John} instructs the \emph{CMS} to process a specific mission.\\
2. \emph{CMS} selects the internal worker \emph{Bob} to execute the mission.\\
3. \emph{CMS} instructs `\emph{Bob} to behave as \emph{FAW}.\\
4. \emph{Bob} informs to the \emph{CMS} of his arrival.\\
5. \emph{Bob} executes the mission.\\
6. \emph{Bob} informs to the \emph{CMS} the mission outcome.


\item [\textbf{Extensions}]:\\
2.a None internal worker can execute the mission.\\
\hspace*{0.5cm} 2.a.1 \emph{CMS} requests an external resource to \emph{ERS}.\\
\hspace*{0.5cm} 2.a.2 \emph{ERS} informs \emph{CMS} that the request can be processed.\\
\hspace*{1.4cm} Use case continues at step 3.

}

\end{lyxlist}
\hrule
\vspace{0.5cm}

\Remark{Graphical User Interfaces (GUIs)}: include GUIs screenshots to show the
different stages of the process while its is performed by the actor.



\section{Actors common procedures}
Common procedures to several actors are grouped in this section to avoid
redundancy.

\subsection{Trigger the alert state}
\vspace{0.5cm}
\hrule
\vspace{0.5cm}
\begin{lyxlist}{UC1}
\small{
\item [\textbf{Use~Case:}] TriggerAlertState
\item [\textbf{Scope:}] Crisis Management System (\emph{CMS})
\item [\textbf{Primary Actor}:] Medical Department, Government
\item [\textbf{Secondary Actor}:] None
\item [\textbf{Intention:}] According to the diagnostic or decision of one of
the primary actors, an alert will will send a notification to SHeavy about a
possible epidemic.
\item [\textbf{Level}:]Subfunctional level
\item [\textbf{Main~Success~Scenario}]:\\
1. Medical Department finds out that an epidemic is possible. They will
immediately send a confirmed notification to SHeavy.\\
2. Military Department finds out that an epidemic is possible. They will
immediately send a confirmed notification to SHeavy.\\
3. The Government finds out that an epidemic is possible. They will
immediately send a confirmed notification to SHeavy.\\
}
\end{lyxlist}
\hrule 
\vspace{0.5cm} 

\subsection{Lift the alert state}
\vspace{0.5cm}
\hrule
\vspace{0.5cm}
\begin{lyxlist}{UC1}
\small{
\item [\textbf{Use~Case:}] LiftAlertState
\item [\textbf{Scope:}] Crisis Management System (\emph{CMS})
\item [\textbf{Primary Actor}:] Medical Department, Government
\item [\textbf{Secondary Actor}:] None
\item [\textbf{Intention:}] According to the diagnostic or decision of one of
the primary actors, an alert will will send a notification to SHeavy about a
possible epidemic.
\item [\textbf{Level}:]Subfunctional level
\item [\textbf{Main~Success~Scenario}]:\\
1. Medical Department notices that the epidemic is over or that the
diagnostic was not right. They will immediately send a notification to SHeavy
to lift the alert.\\
2. Military Department finds out that an epidemic is over or that their
information was not right. They will immediately send a notification to SHeavy
to lift the alert.\\
3. The Government finds out that an epidemic is over or that their
information was not right. They will immediately send a notification to SHeavy
to lift the alert.\\
4. The system will automatically lift the pre-alert if the alert notification
was not confirmed. For example due to a misclick or misfunction of the system.\\
}
\end{lyxlist}
\hrule
\vspace{0.5cm} 

\subsection{Alert Assessements}
\vspace{0.5cm}
\hrule
\vspace{0.5cm}
\begin{lyxlist}{UC1}
\small{
\item [\textbf{Use~Case:}] AlertAssessements
\item [\textbf{Scope:}] Crisis Management System (\emph{CMS})
\item [\textbf{Primary Actor}:] Medical Department, Government
\item [\textbf{Secondary Actor}:] None
\item [\textbf{Intention:}] After the recognition of an possible epidemic, the
concerned primary, which is going to trigger the alert, will rate the epidemic
in a color-scale (orange or red).\\
\item [\textbf{Level}:]Analyse level
\item [\textbf{Main~Success~Scenario}]:\\
1. If there are some cases of an epidemic, the concerned primary actor willl
trigger the orange level alert.\\
2. If the epidemic is already spread out or infected person percentage is
greater than 1/4 of the country's population then the concerned primary actor
will trigger the red level alert.\\
3. The green level alert is triggered per default in other words there's no
epidemic.\\ }
\end{lyxlist}
\hrule
\vspace{0.5cm} 

\subsection{Update Map after Alert State}
\vspace{0.5cm}
\hrule
\vspace{0.5cm}
\begin{lyxlist}{UC1}
\small{
\item [\textbf{Use~Case:}] AlertAssessements
\item [\textbf{Scope:}] Crisis Management System (\emph{CMS})
\item [\textbf{Primary Actor}:] Medical Department, Government
\item [\textbf{Secondary Actor}:] None
\item [\textbf{Intention:}] Update the map and GPS accordingly to 
the confirmation of a change of zones\\
\item [\textbf{Level}:]Sub-functional level
\item [\textbf{Main~Success~Scenario}]:\\
1.	The Medical Department and/or the Military Department uses the TriggerAlertState
 or LiftAlertState and the Government confirms the change of state.\\
2.	SHeavy receives that the state of a city or region is changed to a zone
type.\\
3.	SHeavy updates the map and GPS accordingly to the alert message.\\
4.	SHeavy sent a message to the Medical Department, Military Department and Government 
that the state has changed.\\
5.	SHeavy sent an alert message to the Common User in order to inform this
change.\\
Extensions:\\
	3.a The state of a city or region is set or changed to Safe, Unsafe or
	Danger.\\
		3.a.1 After the TriggerAlertState or LiftAlertState, Sheavy receives the
		state.\\
		3.a.2 SHeavy uses that information to set new zones and update the map and
		GPS.\\
}
\end{lyxlist}
\hrule
\vspace{0.5cm} 

\subsection{Set/Change Zone State}
\vspace{0.5cm}
\hrule
\vspace{0.5cm}
\begin{lyxlist}{UC1}
\small{
\item [\textbf{Use~Case:}] SetZoneState
\item [\textbf{Scope:}] Crisis Management System (\emph{CMS})
\item [\textbf{Primary Actor}:] Medical Department, Government
\item [\textbf{Secondary Actor}:] None
\item [\textbf{Intention:}] Set or change the state of a zone 
after the confirmation of the Government.\\
\item [\textbf{Level}:]Sub-functional level
\item [\textbf{Main~Success~Scenario}]:\\
1. After the TriggerAlertState or LiftAlertState, the Government confirms
 the respective state of a zone: Safe, Unsafe, Danger.\\
2. The state of zone is also received by SHeavy.\\
3. SHeavy sets the state to a new zone or changes the state of a current zone to 
the respective new state.\\
4. SHeavy updates the state of that zone in the database that updates any
corresponding functions of the application.\\
5. SHeavy notifies the Medical Department and the Military Department that a new 
region has the corresponding state or that an existing region changed his state
to the new state.\\
}
\end{lyxlist}
\hrule
\vspace{0.5cm} 

\section{Medical Department procedures}

\subsection{Handling of infected patient}
\vspace{0.5cm}
\hrule
\vspace{0.5cm}
\begin{lyxlist}{UC1}
\small{
\item [\textbf{Use~Case:}] HandlingOfInfectedPatient
\item [\textbf{Scope:}] Crisis Management System (\emph{CMS})
\item [\textbf{Primary Actor}:] Medical Department
\item [\textbf{Secondary Actor}:] None
\item [\textbf{Intention:}]The Medical Department intends to update
the application with the newest data about infected people, no matter what
sickness they have. In case of a known epidemic infection, keep an updated
record of the growth of the epidemic.
\item [\textbf{Level}:]Subfunctional level
\item [\textbf{Main~Success~Scenario}]:\\
1. The Medical Department performs a medical check on a patient.\\
2. The Medical Department sends data about that patient to the application.\\
3. If the patient shows signs of an infection that is an already known epidemic infection, his data
and location will be automatically updated on the data center part dedicated to
this matter and also notify on SHeavy.\\
4. If the patient shows signs of an unknown infection the medical Department
will proceed on taking a sample of blood for further testing.\\
5. The Medical sends the sample to the nearest laboratory in order to find a
cure.\\
}
\end{lyxlist}
\hrule
\vspace{0.5cm} 

\section{Common users procedures}

\subsection{Alert while entering a Danger Zone}
\vspace{0.5cm}
\hrule
\vspace{0.5cm}
\begin{lyxlist}{UC1}
\small{
\item [\textbf{Use~Case:}] DangerAlert
\item [\textbf{Scope:}] Crisis Management System (\emph{CMS})
\item [\textbf{Primary Actor}:] Common user
\item [\textbf{Secondary Actor}:] None
\item [\textbf{Intention:}]Warn a user that he is about to enter a Danger Zone 
before entering the Danger Zone.
\item [\textbf{Level}:]Sub-functional level
\item [\textbf{Main~Success~Scenario}]:\\
1. The common user is going somewhere in or near a Danger Zone to do
something.\\
2. The common user is located by SHeavy and John is near a Danger Zone and is
going in its direction.\\
3. SHeavy immediately send a warning to John's Phone, indicating 
that John is entering a Danger Zone.\\
4. The common user sees the warning and he takes a way around the Danger Zone.\\
}
\end{lyxlist}
\hrule
\vspace{0.5cm} 

\section{System procedures}
\subsection{Finding safe route}
\vspace{0.5cm}
\hrule
\vspace{0.5cm}
\begin{lyxlist}{UC1}
\small{
\item [\textbf{Use~Case:}] FindSafe
\item [\textbf{Scope:}] Crisis Management System (\emph{CMS})
\item [\textbf{Primary Actor}:] System
\item [\textbf{Secondary Actor}:] Common User
\item [\textbf{Intention:}]Suggest the user the safest way to get to a safe
place (e.g. hospital)
\item [\textbf{Level}:]Sub-functional level
\item [\textbf{Main~Success~Scenario}]:\\
1. The common user is going to a safe place and uses the GPS of SHeavy.\\
2. SHeavy calculates the fastest way to the indicated place.\\
3. SHeavy shows the safest way from the GPS to John.\\
Extensions:\\
2.a SHeavy calculates the fastest way avoiding Unsafe and Danger Zones.\\
	2.a.1 SHeavy ignores any way through an Unsafe/Danger Zone.\\
	2.a.2 SHeavy calculates the fastest way from the non-ignored ways.\\
\item 
}
\end{lyxlist}
\hrule
\vspace{0.5cm}  
















